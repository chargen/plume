\documentclass[a4paper,11pt]{article}
\usepackage[T1]{fontenc}
\usepackage[english,francais]{babel}
\usepackage[utf8]{inputenc}
\usepackage{amsmath}
\usepackage{hyperref}
%\usepackage[squaren, Gray, cdot]{SIunits}
\author{PLUME}
\title{Etude}
\begin{document}
\maketitle
\tableofcontents

\section{Physical background}

(p431-pirkl.pdf)

A transmitter generates
an oscillating magnetic field with a well defined narrow
frequency. The field excites a receiver which is essentially
an oscillating circuit precisely tuned to the respective frequency.
The strength of the excitation (=induced voltage)
depends on the location of the receiver in the transmitter
field, which is the basic effect that we use for positioning.

\subsection{The Good News}

From the point of view of indoor localization this has a number
of advantages:
\begin{itemize}
\item Since nothing propagates, there are also no problems with
multi path propagation which is a major issue in RF (and
ultrasonic) systems, in particular (but not only) in time of
flight measurement based approaches
\item As no waves are generated, wave phenomena such as interference
and refraction do not occur. In RF systems such
phenomena are responsible for so called ”fading” which
means that signal intensity displays complex spatial variations
that are much stronger than the distance dependent
signal strength change used for position computation.
\item Magnetic field is much more difficult to ”block” than electromagnetic
waves (at least within the spectrum range usable
for positioning systems). In particular, water, which
is the main component of the human body, has no relevant
influence. Thus, presence of people which is a major
factor in for exampleWiFi positioning, has no negative effect
on a resonant magnetic system. In essence only massive
metallic objects, in particular strongly ferromagnetic
ones, have a significant effect on the signal.
\item Even massive metal objects do not block the signal but
have mostly local influence. While the influence of ferromagnetic
objects on the field can be complex, on an abstract
level it can be described as bending the field lines
running within the object. To match the bending inside
the object the field lines outside are also deformed but to
a lesser degree and mostly in the proximity of the object.
Thus we only see an effect if the object is close to the receiver
or (to a lesser extent) to the transmitter. Unlike with
RF systems the mere presence of the object in the vicinity
has virtually no effect. This means that there is no need
for finger printing to account for the specific configuration
of furniture, walls etc.
\item The magnetic field of a typical coil is elongated along the
coil axis so that the field strength depends not only on the
distance but also on where around the coil the receiver is
placed. As will be discussed in section Position Estimation
this means that using three perpendicular transmitter
coils placed at a single location allows the receiver position
to be restricted to a few points given by the symmetry
of the magnetic field. Exact 3D positioning is possible
with two beacon nodes. This is a major difference to other
beacon based technologies where at least three beacons
placed at different locations are needed for an accurate
estimate and a single beacon only allows to narrow down
the position to a circle rather than a small set of points.
\item The strength of the signal induced in the receiver coil depends
not only on the position within the field of the transmitter,
but also on the angle between the receiver coil and
the field lines. This means that the signal contains not
only position, but also orientation information.
\end{itemize}

In summary, because of the underlying physical principle,
magnetic resonant coupling technology is much less sensitive
to disturbance typically found in indoor environments.
Because of the structure of the field a single beacon only
is needed for a basic position estimate and the signal even
contains information about receiver orientation.

\subsection{The Bad News}

The physical principle of resonant magnetic coupling also
creates some problems that need to be addressed by the positioning
system design:
\begin{itemize}
\item The energy transmitted in the field decreases with 1=r6
which means that the system is inherently short range.
The range depends on the size of the coil, the number of
windings and the voltage of the transmitter. Given that
the transmitters are stationary (which means that size is
not that critical) and are operated from the mains (which
means large voltages are feasible) we have been able to
achieve a range of approx. 4m (covers a diameter of 8m)
so that a single transmitter can in principle cover an area of
approx. 50 m2 (single large room). However this comes
at the price of having to deal with a large dynamic range
which complicates the design of the receiver.
\item The receiver has a very narrow resonant range which is
good in terms of being insensitive to noise from other
sources of magnetic field. On the other hand it means
that any variations in the transmission frequency or the
resonant frequency have a critical impact on the measured
signal. Such variations are common due to component
manufacturing tolerances of the electronic components as
well as from temperature effects and ways must be found
to deal with them.
\item The complex structure of the magnetic field means that
translation from the signal strength to position is not trivial
and involves detailed modeling of the field.
\item Because of the orientation dependence of the signal not
only the transmitter, but also the receiver needs to have
three perpendicular coils and we need to measure the signal
from each transmitter coil on each receiver coil. To do
this we need time synchronization between the receivers
and the transmitters. Such synchronization must allow
for multiple transmitters to be operating in the same area,
overlapping in their fields.
\end{itemize}

\subsection{Health and Safety Considerations}
Health and safety considerations are an important aspect of
any practical system. In our case the obvious concern are
the generated magnetic field strengths.
The WHO guidance for safe field
strengths of low frequency magnetic fields are bellow 100 $\mu T$ .
Hair dryers emit between 1.5 mT and 2 mT at a distance of 3cm (which is how they
are used).

Source : WHO chapter 12 \url{http://www.who.int/peh-emf/publications/elf_ehc/en/}

\section{Physique}

\subsection{Loi de Faraday}

$ e = - \frac{d\Phi}{dt} $

Flux magnétique $ {\textrm  {d}}\Phi ={\vec  {B}}\cdot {\vec  {{\textrm  {d}}S}}=\|{\vec  B}\|\cdot \|{\vec  {{\textrm  {d}}S}}\|\cdot \cos \theta  $

\subsection{Effet de peau}

$ \delta = \frac {1} {\sqrt{\sigma\cdot\mu\cdot\pi\cdot f}} $

Pour le cuivre à 20 kHz $ \delta = 0.46 mm $

\subsection{Fil de cuivre}

Résitance $ R=\rho {\frac  {L}{S}} $

Puissance dissipé par effet Joule $ P=\rho {\frac  {L}{S}}I^{{2}} $ 

Résistivité du cuivre a 300K $ \rho = 17 * 10^{-9} $

\subsection{Bobine}

$ B(\phi, r) = \frac{\mu_0 INS}{4\pi r^4}\sqrt{1 + 3sin\phi^2} $

$ \mu_0 = 4\pi*10^{-7} $

La loi de Biot et Savart est valide pour un courant variable tant qu'on peut négliger le temps mis par la lumière pour aller d'un point quelconque de la bobine au point d'observation devant la période du courant.

Temps mis par la lumière pour parcourir 5m : 1.66782048e-8 s.
Il faut 1.66782048e-8 << T.
Donc la fréquence limite de validité de la formule est f << 60 MHz.

\subsection{Noise}

\url{http://cds.linear.com/docs/en/datasheet/1028fb.pdf}
\url{http://www.ti.com/lit/ml/sloa082/sloa082.pdf}

\subsection{Fourier}

\url{http://perso.telecom-paristech.fr/~prado/recherche/fourier/tr_fourier.pdf}

\section{Calcul de la position}

\subsection{Modélisation du problème et notations}

Repères :
\begin{itemize}
\item cartésien fixe de l'emetteur : $R = (O,e_x,e_y,e_z)$
\item sphérique fixe de l'emetteur : $R = (O,e_r,e_\theta,e_\phi)$
\item mobile du recepteur : $R' = (M,e_{x'},e_{y'},e_{z'})$
\end{itemize}

L'emetteur est constitué de trois bobines centrées en O et alignées selon les axes $(O,e_x)$,$(O,e_y)$ et $(O,e_z)$

Les champs magnétiques produits par ces bobines sont notés en coordonnées sphériques et dans le repère du recepteur :

$\vec{B^{(x)}}
=
\left(
  \begin{array}{ c }
    B^{(x)}_{r} \\
    B^{(x)}_{\theta} \\
    B^{(x)}_{\phi}
  \end{array} \right)
=
\left(
  \begin{array}{ c }
    B^{(x)}_{x'} \\
    B^{(x)}_{y'} \\
    B^{(x)}_{z'}
  \end{array} \right)$

$\vec{B^{(y)}}
=
\left(
  \begin{array}{ c }
    B^{(y)}_{r} \\
    B^{(y)}_{\theta} \\
    B^{(y)}_{\phi}
  \end{array} \right)
=
\left(
  \begin{array}{ c }
    B^{(y)}_{x'} \\
    B^{(y)}_{y'} \\
    B^{(y)}_{z'}
  \end{array} \right)$ 

$\vec{B^{(z)}}
=
\left(
  \begin{array}{ c }
    B^{(z)}_{r} \\
    B^{(z)}_{\theta} \\
    B^{(z)}_{\phi}
  \end{array} \right)
=
\left(
  \begin{array}{ c }
    B^{(z)}_{x'} \\
    B^{(z)}_{y'} \\
    B^{(z)}_{z'}
  \end{array} \right)$

Ils sont obtenus à partir des tensions resultant de l'induction dans les bobines orthogonales de vecteur surface $\vec(S_{x'}) = S_{x'}\cdot \vec{e_{x'}}$ :
\[\vec{V^{(i)}}
=
\left(
  \begin{array}{ c }
    V^{(i)}_{x'} \\
    V^{(i)}_{y'} \\
    V^{(i)}_{z'}
  \end{array} 
\right)
=
\left(
  \begin{array}{ c }
    \frac{\partial\vec{B^{(i)}}}{\partial t} \cdot \vec{S_{x'}} \\
    \frac{\partial\vec{B^{(i)}}}{\partial t} \cdot \vec{S_{y'}} \\
    \frac{\partial\vec{B^{(i)}}}{\partial t} \cdot \vec{S_{z'}} 
  \end{array} 
\right)
=
\left(
  \begin{array}{ c }
    \vec{S_{x'}} \\
    \vec{S_{y'}} \\
    \vec{S_{z'}} 
  \end{array} 
\right)
\cdot
\frac{\partial\vec{B^{(i)}}}{\partial t}
=
\left(
  \begin{array}{ c }
    \vec{S_{x'}} \\
    \vec{S_{y'}} \\
    \vec{S_{z'}} 
  \end{array} 
\right)
\cdot
j \omega^{(i)} \vec{B^{(i)}}
\]

Si la bobine alignée sur l'axe $(O,\vec{e_i})$ est parcourue par un courant alternatif sinusoïdal de période $\omega^{(i)}$.

\subsection{Calcul de la position}

Les bobines du récepteurs étant orthogonales, on peut calculer les vecteurs $\vec{B^{(i)}}$ entièrement.

On obtient notamment leurs normes au point M : 
\begin{align}
  \left|B^{(i)}\right|^2 &= \left| \vec{B^{(i)}}\cdot \vec{e_{x'}}\right|^2 + 
                           \left| \vec{B^{(i)}}\cdot \vec{e_{y'}}\right|^2 + 
                           \left| \vec{B^{(i)}}\cdot \vec{e_{z'}}\right|^2\\
                        &= \left| \frac{V^{(i)}_{x'}}{S_{x'}\omega^{(i)}}\right|^2 + 
                           \left| \frac{V^{(i)}_{y'}}{S_{y'}\omega^{(i)}}\right|^2 + 
                           \left| \frac{V^{(i)}_{z'}}{S_{z'}\omega^{(i)}}\right|^2
\end{align}

Or on sait que le champ en le point M vaut en coordonnées sphériques :
\[
\vec{B^{(z)}}
=
\left(
  \begin{array}{ c }
    B^{(z)}_{r} \\
    B^{(z)}_{\theta} \\
    B^{(z)}_{\phi}
  \end{array} \right)
=
\left(
  \begin{array}{ c }
    \frac{\mu_0}{4\pi}\frac{2M^{(z)} cos(\theta)}{r^3} \\
    \frac{\mu_0}{4\pi}\frac{M^{(z)} sin(\theta)}{r^3} \\
    0
  \end{array} \right)
=
\left(
  \begin{array}{ c }
    \frac{2G^{(z)} cos(\theta)}{r^3} \\
    \frac{G^{(z)} sin(\theta)}{r^3} \\
    0
  \end{array} \right)
\]

En notant $G^{(i)}$ le gain $\frac{\mu_0}{4\pi}M^{(i)}$ de la bobine émettrice orientée selon l'axe $(O,\vec{e_i})$.

Ainsi :
\begin{align}
\left|B^{(i)}\right|^2 &= \left(G^{(i)}\right)^2\frac{1}{r^6}\left(sin(\theta)^2 + 4 cos(\theta)^2\right) \\
&= \left(G^{(i)}\right)^2\frac{1}{r^6}\left(1 + 3 cos(\theta)^2\right) \\
&= \left(G^{(i)}\right)^2\frac{1}{r^6}\left(1 + 3 \left(\frac{\vec{OM}\cdot\vec{e_i}}{|\vec{OM}|}\right)^2\right)
\end{align}

Soit :

\[
\left|B^{(x)}\right|^2 = \left(G^{(x)}\right)^2\frac{1}{r^6}\left(1 + 3 \frac{x^2}{r^2}\right)
\]
\[
\left|B^{(y)}\right|^2 = \left(G^{(y)}\right)^2\frac{1}{r^6}\left(1 + 3 \frac{y^2}{r^2}\right)
\]
\[
\left|B^{(z)}\right|^2 = \left(G^{(z)}\right)^2\frac{1}{r^6}\left(1 + 3 \frac{z^2}{r^2}\right)
\]

On a donc : 

\begin{align}
\frac{\left|B^{(x)}\right|^2}{\left(G^{(x)}\right)^2} + 
\frac{\left|B^{(y)}\right|^2}{\left(G^{(y)}\right)^2} + 
\frac{\left|B^{(z)}\right|^2}{\left(G^{(z)}\right)^2} &= \frac{1}{r^6}\left(3 + 3 \frac{r^2}{r^2}\right)\\
&= \frac{6}{r^6}
\end{align}

D'où
\[
r = \sqrt[6]{
  \frac{6}{
    \frac{\left|B^{(x)}\right|^2}{\left(G^{(x)}\right)^2} + 
    \frac{\left|B^{(y)}\right|^2}{\left(G^{(y)}\right)^2} + 
    \frac{\left|B^{(z)}\right|^2}{\left(G^{(z)}\right)^2}
  }
} 
\]
\begingroup
\tiny
\[  
= \sqrt[6]{
  \frac{6}{
    \left| \frac{V^{(x)}_{x'}}{S_{x'}\omega^{(x)} G^{(x)}}\right|^2 + 
    \left| \frac{V^{(x)}_{y'}}{S_{y'}\omega^{(x)} G^{(x)}}\right|^2 + 
    \left| \frac{V^{(x)}_{z'}}{S_{z'}\omega^{(x)} G^{(x)}}\right|^2
    +
    \left| \frac{V^{(y)}_{x'}}{S_{x'}\omega^{(y)} G^{(y)}}\right|^2 + 
    \left| \frac{V^{(y)}_{y'}}{S_{y'}\omega^{(y)} G^{(y)}}\right|^2 + 
    \left| \frac{V^{(y)}_{z'}}{S_{z'}\omega^{(y)} G^{(y)}}\right|^2
    +
    \left| \frac{V^{(z)}_{x'}}{S_{x'}\omega^{(z)} G^{(z)}}\right|^2 + 
    \left| \frac{V^{(z)}_{y'}}{S_{y'}\omega^{(z)} G^{(z)}}\right|^2 + 
    \left| \frac{V^{(z)}_{z'}}{S_{z'}\omega^{(z)} G^{(z)}}\right|^2
  }
}
\]
\[
= \sqrt[6]{
  \frac{6}{
    \left|\left|
        \left(
          \begin{array}{l c r}
            G^{(x)} & 0 & 0 \\
            0 & G^{(y)}& 0 \\
            0 & 0 & G^{(z)}
          \end{array} 
        \right)^{-1}
        \cdot
        \left(
          \begin{array}{l c r}
            \omega^{(x)} & 0 & 0 \\
            0 & \omega^{(y)}& 0 \\
            0 & 0 & \omega^{(z)}
          \end{array} 
        \right)^{-1}
        \cdot
        \left(
          \begin{array}{l c r}
            V^{(x)}_{x'} & V^{(y)}_{x'} & V^{(z)}_{x'} \\
            V^{(x)}_{y'} & V^{(y)}_{y'} & V^{(z)}_{y'} \\
            V^{(x)}_{z'} & V^{(y)}_{z'} & V^{(z)}_{z'} 
          \end{array} 
        \right)
        \cdot
        \left(
          \begin{array}{l c r}
            S_{x'} & 0 & 0 \\
            0 & S_{y'} & 0 \\
            0 & 0 & S_{z'}
          \end{array}
        \right)^{-1}
      \right|\right|^2
  }
}
\]
\endgroup
\[
= \sqrt[6]{
  \frac{6}{
    \left|\left|
        \mathbf{G}
        \cdot
        \mathbf{\Omega}
        \cdot
        \mathbf{V}
        \cdot
        \mathbf{S}
      \right|\right|^2
  }
} 
\]

Cette formule correspond à un cas idéal, cependant si l'on doit considérer des interférences ou des inductions mutuelles entre les bobines d'émission ou entre les bobines de reception, il suffira de modifier les matrices $\mathbf{G}$ et $\mathbf{S}$ pour ajouter des termes non nuls en dehors de leur diagonale. Cela permettra d'obtenir des champs magnétiques normalisés $\frac{\Vec{B^{(i)}}}{G^{(i)}}$à partir des valeurs de tension mesurées.

On pourra de plus s'astreindre de la multiplication par $\mathbf{\Omega}$ en utilisant un circuit intégrateur.
\newpage

\section{Dimensionnement des bobines}

cf MATLAB

\end{document}